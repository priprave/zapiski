\naslov{Dodatni formatniki za \texttt{scanf}}

Poleg že znanih formatnikov \verb+%d+, \verb+%lld+ in \verb+%s+ poznamo tudi
druge, ki so lahko uporabni v različnih situacijah.
Seznam pogosto uporabnih formatnikov je v
tabeli~\ref{tab:vhod-in-izhod-dopolnitev:formatniki}.
V tabeli se pojavi izraz \emph{prazen znak}, angl.~\emph{whitespace} --- to je
znak, ki zasede prostor v spominu (in kateremu pripada ASCII koda), vendar se na
zaslonu ne pojavi.
Poznamo tri take znake, to se presledek, tabulator \verb+\t+ in znak za novo
vrstico \verb+\n+.
Formatniki \verb+%d+, \verb+%lld+ in \verb+%s+ preberejo, a ignorirajo, prazne
znake, ki se pojavijo pred vsebino, ki jo formatnik dejansko shrani.
Če npr.~v spodnji primer programa vpišemo vhod
\begin{verbatim*}
Hello

   World!
\end{verbatim*}
bo program prebral le ti dve besedi, in ne praznih znakov med njima.

\begin{table}[h]
  \centering
  \begin{tabular}{c|l}
	formatnik & opis \\
	\hline
	\verb+%%+ & en znak \verb+%+ \\
	\verb+%d+ & število tipa \koda{int} \\
	\verb+%lld+ & število tipa \koda{long long} \\
	\verb+%c+ & poljuben znak (lahko tudi prazen znak) \\
	\verb+%s+ & zaporedje nepraznih znakov \\
	\verb+%[...]+ & zaporedje znakov, ki se pojavijo med oglatimi oklepaji \\
	\verb+%[^...]+ & zaporedje znakov, ki se ne pojavijo med oglatimi oklepaji
  \end{tabular}
  \caption{Različni formatniki za \koda{scanf} in \koda{printf}}%
  \label{tab:vhod-in-izhod-dopolnitev:formatniki}
\end{table}


\fkoda{poglavja/vhod-in-izhod-dopolnitev/formatnik-s.cpp}

Zadnja formatnika v tabeli, \verb+%[...]+ in \verb+%[^...]+ sta bolj zapletena
od ostalih.
Uporabljamo ju tako, da namesto tropičja v oglate oklepaje zapišemo seznam
znakov, ki so dovoljeni v prebranem besedilu.
Tako lahko npr.~enostavno ločimo besedo, sestavljeno iz črk in številk na niz in
na število, ki ga lahko takoj shranimo v \koda{int}:

\fkoda{poglavja/vhod-in-izhod-dopolnitev/primer-branje-niza-in-stevila.cpp}

Če zgornjemu programu kot vhod podamo \texttt{aacbb012}, bo izpisal
\begin{verbatim}
Niz: aacbb, stevilo: 12
\end{verbatim}
Če želimo dovoliti vse male črke abecede, lahko namesto seznama vseh črk
zapišemo tudi \verb+%[a-z]+, podobno lahko za velike črke uporabimo
\verb+%[A-Z]+, za števke pa \verb+%[0-9]+.
Formatnik bo v zadnjem primeru še vedno prebral niz znakov, in ga ne bo
avtomatsko pretvoril v \koda{int}.
Tudi te obsege lahko kombiniramo, formatnik \verb+%[a-z2-7B]+ npr.~prebere niz,
sestavljen iz malih črk angleške abecede, števk med $2$ in $7$ ter velike črke
B.

Formatnik \verb+%[^...]+ deluje podobno kot formatnik \verb+%[...]+, le da
prebere vse znake, razen tistih, ki smo jih zapisali namesto tropičja.
Ta formatnik je uporaben pri branju niza do konca vrstice, česar s formatnikom
\verb+%s+ ne moremo narediti, če nimamo podanega števila besed v vrstici.
Primer je prikazan v spodnjem programu, kjer uporabimo tudi formatnik
\verb+%*c+.
Ta deluje enako kot formatnik \verb+%c+, le da znaka, ki ga prebere, nikjer ne
shrani, zato zanj tudi ne podamo spremenljivke.
Ta formatnik je potreben, ker \verb+%[^\n]+ ne prebere znaka za novo vrstico, ki
ga \enquote{počistimo} z \verb+%*c+.
Če bi želeli prebrati dve zaporedni vrstici in nebi uporabili \verb+%*c+, nebi
druga uporaba formatnika \verb+%[^\n]+ shranila ničesar, saj bi nemudoma
naletela na znak za novo vrstico, in zaključila branje.

\fkoda{poglavja/vhod-in-izhod-dopolnitev/do-konca-vrstice.cpp}

\naslov{Neznano število podatkov}

Če vemo točno, koliko besed, številk oz.~vrstic bomo imeli na vhodu, jih lahko
preberemo z zanko.
Kaj pa, če ne vemo, koliko podatkov bo na vhodu, kot v nalogi
\putka{neznane-vsote}{Neznane vsote}?
Če podatke beremo v neskončni zanki, bomo sicer prebrali vse, a se program ne
bo nikoli ustavil.
V naslednjem primeru je nakazano, kako ta problem rešimo:

\fkoda{poglavja/vhod-in-izhod-dopolnitev/do-konca-vhoda.cpp}

Funkcija \koda{scanf} vrne število formatnikov, ki jih je uspešno prebrala (če
ji kot parameter npr.~podamo samo \verb+%d+, bo vrnila $1$, če je uspešno
prebrala število, sicer pa $0$).
Posebno vrednost \koda{EOF} (End of File) pa vrne, če na vhodu ni ničesar več za
prebrati.
Tedaj se bo zanka ustavila.
Če programu vhodne podatke podajamo iz datoteke, se to zgodi avtomatsko ob koncu
datoteke, če pa mu podatke podajamo na roko, konec vhoda sporočimo s
\verb|Ctrl+D| (Linux in MacOS) ali \verb|Ctrl+Z| (Windows).

\naslov{Uporaba nizov ali datotek za vhodno-izhodni sistem}

Včasih si lahko pri procesiranju podatkov pomagamo s tem, da dano besedilo prvo
shranimo v niz, preden ga pretvorimo v drugo obliko, ali iz njega izluščimo
podatke.
Pri tem si lahko pomagamo s funkcijama \koda{sscanf} in \koda{ssprintf}, ki
delujeta podobno kot poznana \koda{scanf} in \koda{printf}, vendar namesto
branja iz oz.~pisanja na standardni vhod oz.~izhod delujeta z nizi v spominu.
Niz, s katerega beremo oz.~na katerega pišemo, podamo kot prvi argument v ti
funkciji, kakor v spodnjem primeru.

\fkoda{poglavja/vhod-in-izhod-dopolnitev/sscanf-sprintf.cpp}

Podobno lahko s pomočjo funkcij \koda{fscanf} ter \koda{fprintf} beremo iz in
pišemo v datoteke na računalniku.
Datoteke predstavimo s posebnim tipom \koda{FILE}, pri katerem moramo poleg
imena spremenljivke napisati tudi zvezdico (zakaj je temu tako, spoznamo v enem
od kasnejših poglavij).
Da se povežemo s pravo datoteko, uporabimo funkcijo \koda{fopen}, ki ji podamo
dva argumenta.
Prvi argument je ime datoteke, ki jo želimo uporabljati, drug argument pa je
možnost odpiranja.
Za naše potrebe sta dovolj dve možnosti, \koda{"r"}, ki pove, da bomo datoteko
odprli v načinu za branje, ter \koda{"w"}, ki pove, da bomo datoteko odprli v
načinu za pisanje.
Slednja možnost tudi ustvari datoteko z danim imenom, če ta ne obstaja, oziroma
izbriše vso besedilo v tej datoteki, če datoteka že ima vsebino.
Primer uporabe funkcij \koda{fscanf} in \koda{fprintf} je prikazan spodaj.
Opazimo, da se na koncu programa pojavi tudi funkcija \koda{fclose}.
Ta funkcija operacijskemu sistemu pove, da smo z branjem oz.~pisanjem
zaključili, in da lahko spremembe na tej točki shrani v dejansko datoteko (brez
klica te funkcije shranjevanje sprememb ni zagotovljeno).
Funkcijo moramo nujno poklicati za vsako datoteko, ki smo jo odprli z
\koda{fopen}.

\fkoda{poglavja/vhod-in-izhod-dopolnitev/datoteke.cpp}

%#endblock
% LocalWords:   formatniki scanf formatnikov formatnik formatnika tropičja
% LocalWords:  formatnikom vhodno-izhodni
