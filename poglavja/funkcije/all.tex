\naslov{Osnovna sintaksa}

Ko pišemo daljše programe, se pogosto zgodi, da večkrat uporabimo nek podoben
kos kode.
Da si v takih situacijah olajšamo življenje, lahko uporabimo \emph{funkcije}.
Funkciji podamo nekaj \emph{parametrov} (včasih se uporabi tudi beseda
\emph{argumentov}), ona nekaj melje in nam potem \emph{vrne} neko vrednost.
Poleg tega, da samo vrne neko vrednost, ima lahko funkcija tudi kakšne stranske
učinke.
Lahko na primer kaj izpiše z uporabo \verb+printf+, ali pa spremeni vrednost
neke spremenljivke.
Včasih so parametri ali vrnjena vrednost nepotrebni, zato lahko napišemo tudi
funkcije brez parametrov ali funkcije, ki ne vrnejo ničesar.

Najbolj osnovna oblika funkcije je funkcija, ki ne sprejme nobenega parametra in
ne vrne ničesar.
Če hočemo, da ta funkcija ni neuporabna, bo morala imeti kakšen stranski učinek.
V spodnjem primeru funkcija izpiše besedilo.

\fkoda{poglavja/funkcije/osnovna-funkcija.cpp}

Poglejmo si malo bolj podrobno, kaj posamični deli zgornjega programa naredijo.
Pri definiciji funkcije smo prvo napisali \koda{void} --- to je posebna oznaka,
ki pomeni, da funkcija ne vrne ničesar.
Za tem je beseda \koda{moja_funkcija}, ki označuje ime funkcije, ki jo
definiramo.
Tako kot spremenljivke lahko funkcije poimenujemo, kakor želimo.
Za imenom funkcije je običajen oklepaj, ki zaznamuje začetek seznama argumentov.
V tem primeru argumentov nimamo, zato seznam takoj končamo z zaklepajem.
Na koncu še v zavite oklepaje postavimo kodo, ki jo bo funkcija izvedla.

Če želimo funkciji podati argumente, jih zapišemo kot spremenljivke med oklepaje
v definiciji funkcije:

\fkoda{poglavja/funkcije/funkcija-parametri.cpp}

Zgoraj smo v oklepaje za imenom spremenljivke postavili \koda{int n}.
Beseda \koda{int} pove, da je ta parameter celo število, \koda{n} pa je ime za
parameter.
Ime, ki smo si ga izbrali za parameter, uporabljamo v telesu funkcije, da
dostopamo do vrednosti v parametru.

Če je naš parameter seznam, to napišemo kot spodaj.

\fkoda{poglavja/funkcije/funkcija-arr-parameter.cpp}

Da funkcija vrne neko vrednost, moramo namesto \koda{void} napisati tip
vrednosti, ki jo vrne, nato pa nekje v notranjosti funkcije uporabiti
\koda{return}.
Poglejmo si enostaven primer, kjer s funkcijo izračunamo vsoto dveh števil.

\fkoda{poglavja/funkcije/funkcija-return.cpp}

Kot smo navajeni, se ukazi v telesu funkcije izvajajo po vrsti od zgoraj
navzdol.
V njej lahko uporabimo vse, kar smo se do zdaj naučili (lahko ustvarjamo nove
spremenljivke, uporabimo pogojne stavke, zanke, \ldots).
Vredno omembe je, da izvajanje funkcije konča takoj, ko se izvede prvi
\koda{return}, tudi če je za tem še nekaj kode.

V spodnjem bloku kode funkcije vrne logično vrednost \koda{bool}, ki je ena od
\koda{true} ali \koda{false}.
Funkcija deluje pravilno, saj če bo \koda{crka} enaka \koda{'a'}, \koda{'e'},
\koda{'i'}, \koda{'o'} ali \koda{'u'}, bo funkcija končala znotraj pogojnega
stavka, sicer pa bo nadaljevala do \koda{return false} na koncu.

\fkoda{poglavja/funkcije/return-konca-funkcijo.cpp}

\naslov{Spremenljivke in funkcije}

Spomnimo se, da lahko spremenljivke definiramo na dva načina.
Tistim, ki so definirane na vrhu, izven funkcij, pravimo \emph{globalne
  spremenljivke}, tistim, ki so pa definirane v neki funkciji, pa \emph{lokalne
  spremenljivke}.
Globalne spremenljivke so vidne povsod, torej v vseh funkcijah, medtem ko lahko
lokalne spremenljivke uporabljamo samo v funkciji, v kateri smo jih definirali.

\fkoda{poglavja/funkcije/globalne-lokalne-spremenljivke.cpp}

Seveda, če globalno spremenljivko nekje spremenimo, se bo spremenila tudi za
vse druge funkcije.
V zgornjem primeru \koda{funkcija} najprej izpiše \koda{g} ($12$) in ga
nato poveča na $15$.
Ko se \koda{g} izpiše drugič, je zato enak $15$.

Če funkciji podamo neko spremenljivko kot parameter in poskušamo to
spremenljivko v funkciji spreminjati, se obnaša na dva različna načina, odvisno
od tega, ali je spremenljivka seznam ali ne.

Če spremenljivka ni seznam (npr.~\koda{int} ali \koda{char}) se za funkcijo
ustvari kopija te spremenljivke.
Če jo v funkciji spremenimo, to ne bo spremenilo izvirne spremenljivke, saj v
funkciji delamo s kopijo.
Če pa argument je seznam, bodo spremembe na seznam v funkciji vplivale na
izvirni seznam, saj se ta ne bo prekopiral.
V spodnjem primeru sta prikazani obe možnosti.

\fkoda{poglavja/funkcije/seznam-se-ne-kopira.cpp}

Sprva ne vidimo smisla, za drugačno obravnavo seznamov, a se izkaže, da je
lahko ta lastnost tudi uporabna.
Če bi želeli napisati funkcijo, ki sprejeme seznam in njegovo dolžino, potem pa
obrne vrstni red elementov tega seznama, lahko to napišemo takole:

\fkoda{poglavja/funkcije/obrni-seznam.cpp}

\naslov{Rekurzija}

\emph{Rekurzija} je način pisanja programa na način, da funkcija kliče samo
sebe.
Pogosto lahko zanko nadomestimo z rekurzijo, če nam je tak način programiranja
ljubši.
Recimo, da želimo sešteti vsa števila med $1$ in nekim $n$.
Z zanko bi to naredili tako, da si pripravimo spremenljivko \koda{vsota}, potem
pa vanjo prištevamo števila, ki jih vidimo v zanki.
Vsoto prvih $n$ števil pa izračuna tudi naslednja funkcija:

\fkoda{poglavja/funkcije/vsota-rekurzivno.cpp}

Najprej napišemo \emph{osnovni pogoj}, ki v primeru, da je argument \koda{n}
manjši ali enak $0$, takoj vrne $0$.
Ta pogoj nujno potrebujemo, saj se bo sicer rekurzija ponavljala v neskončnost.
Po osnovnem pogoju zapišemo \emph{korak rekurzije}, ki izračuna \koda{vsota(n)}
s pomočjo vrednosti funkcije v nekem manjšem številu, v našem primeru kar
\koda{vsota(n - 1)}.

Poglejmo si, kaj se zgodi, če pokličemo \koda{vsota(4)}.
Funkcija izračuna $4 + \operatorname{vsota}(3)$, ampak da bi izračunali to,
moramo prvo izračunati \koda{vsota(3)}.
Pri računanju \koda{vsota(3)} izračunamo $3 + \operatorname{vsota}(2)$, za kar
moramo izračunati \koda{vsota(2)}.
Potem podobno nadaljujejo, pri računanju \koda{vsota(2)} računamo
\koda{vsota(1)}, pri računanju tega pa izračunamo \koda{vsota(0)}.
Tu se rekurzija ustavi, saj pridemo v prvi pogojni stavek, in takoj vrnemo
vrednost $0$.
Potem lahko izračunamo $\operatorname{vsota}(1) = 1 + 0$, iz česar dobimo
$\operatorname{vsota}(2) = 2 + 1$, potem $\operatorname{vsota}(3) = 3 + 2$,
$\operatorname{vsota}(4) = 4 + 6$ in na koncu $\operatorname{vsota}(5) = 5 + 10
= 15$.