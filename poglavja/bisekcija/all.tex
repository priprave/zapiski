\naslov{Osnovna oblika}

Recimo, da imamo dano število $k$, za katerega želimo čim hitreje preveriti, če
se pojavlja v seznamu \koda{arr} dolžine $n$.
Običajen algoritem za tako iskanje preveri vsako število v seznamu, dokler
števila $k$ ne najde, oziroma dokler ne pride do konca seznama.
V najslabšem primeru, če števila $k$ ni v seznamu, ali če se pojavi le na
zadnjem mestu, bo algoritem porabil $O(n)$ operacij za tako iskanje.
To je najbolje možno, če število iščemo le enkrat, če pa v seznamu iščemo veliko
različnih števil, lahko uporabimo spodnji postopek, da iskanje pohitrimo.

Najprej bomo seznam uredili od najmanjšega do največjega števila z enim od
hitrih, t.j.~$O(n \log n)$ algoritmov za urejanje.
V urejenem seznamu lahko iskanje izvajamo na pametnejši način, ki mu pravimo
\emph{bisekcija} ali \emph{binarno iskanje}.
Prvo bomo pogledali število na sredini seznama.
Če je to število večje od $k$, vemo, da se $k$ bodisi nahaja na levi
polovici seznama, bodisi ga na seznamu sploh ni -- če bi se namreč $k$ pojavil
na desni polovici seznama, potem seznam nebi bil urejen!
Podobno, če je element na sredini seznama manjši od $k$, se $k$ nahaja
v desni polovici seznama, ali pa ga v seznamu ni.
S tem enim pogledom smo zmanjšali število možnih elementov, ki jih moramo
pregledati, iz $n$ na $\frac{n}{2}$.
Če postopek ponovimo na polovičnemu seznamu, torej ga spet razdelimo na
polovico, nato dobljeno četrtino še enkrat razdelimo na polovico, in tako dalje,
potem smo pogledali le $\log n$ elementov, da smo našli število $k$, oziroma
ugotovili, da ga v seznamu ni.

Iskanje lahko implementiramo, kot je napisano spodaj:
\fkoda{poglavja/bisekcija/bisekcija.cpp}
V postopku definiramo spremenljivki \koda{levo} in \koda{desno}, ki označujeta
levi in desni rob dela seznama, ki ga moramo pregledati.
V vsaki iteraciji izračunamo sredino seznama, in preverimo, če smo slučajno
naleteli na število $k$.
Če smo, lahko takoj končamo, sicer pa glede na vrednost \koda{arr[sredina]}
primerno spremenimo levi oziroma desni rob iskanja.
V primeru \koda{arr[sredina] > k} vemo, da je število na sredini trenutnega
podseznama preveliko, torej se $k$ pojavlja na levi polovici, vemo pa tudi, da
se ne pojavi na indeksu \koda{sredina}, zato lahko nastavimo novo desno mejo na
\koda{sredina-1}.
Podobno v drugem primeru, če je \koda{arr[sredina] < k}, nastavimo novo levo
mejo na \koda{sredina+1}.

Če bi si shranili tudi indeks, kjer smo našli število $k$, bi lahko s pomočjo
tega indeksa izračunali, koliko elementov v seznamu je večjih in koliko
elementov seznama je manjših od $k$.
Pri tem moramo paziti, da bisekcija vrne enega od indeksov elementov, ki so
enaki $k$; če se to število v seznamu pojavi več kot enkrat, ni
zagotovljeno, da bodo dobili prvo pojavitev (ali zadnjo, ali sredinsko).

Zgornji program ima časovno zahtevnost $O(n \log n + q \log n)$, ker za iskanje
vsakega elementa potrebujemo $\log n$ korakov.
Običajen postopek bi za isto delo potreboval $O(qn)$ operacij, kar je v primeru,
da sta tako $n$ in $q$ velika, hudo prepočasno.

\naslov{Implementacija v standardni knjižnici}

Bisekcija je kratek algoritem, ki ga lahko hitro spišemo (v zgornjem bloku kode
na primer porabi le $12$ vrstic kode, če ne štejemo komentarjev in praznih
vrstic), včasih pa je spisati cel algoritem zamudno, hkrati pa se lahko hitro
zmotimo.
K sreči se lahko v veliko primerih izognemo pisanju algoritma, in namesto tega
uporabimo implementacijo iz standardne knjižnice C++.
Konkretno sta nam na voljo funkciji \koda{lower_bound} in \koda{upper_bound}, ki
jima podamo kazalca na začetek in konec seznama ter iskano število, funkciji pa
potem na seznamu poiščeta dano število.
Obe imata časovno zahtevnost $O(\log n)$, prav tako pa za obe potrebujemo
vključiti knjižnico \koda{<algorithm>}.

Funkcija \koda{lower_bound} vrne kazalec na prvi element seznama, ki je večji
ali enak iskanemu številu, funkcija \koda{upper_bound} pa vrne kazalec na prvi
element, ki je strogo večji kot iskano število.
Če funkciji ne najdeta takšnih števil, namesto tega vrneta kazalec na konec
seznama.
Poglejmo si, kako bi zgornji primer rešili z uporabo funkcije
\koda{lower_bound}.

\fkoda{poglavja/bisekcija/knjiznicna-implementacija.cpp}

\naslov{Bisekcija po odgovoru}

Namesto iskanja števila v seznamu nam bisekcija lahko poda tudi argument
funkcije, pri katerem funkcija da določeno vrednost.
To metodo reševanja imenujemo \emph{bisekcija po odgovoru} oziroma angleško
\emph{binary search the answer}.

Če želimo poiskati \koda{x}, pri katerem je \koda{f(x) == k}, lahko v zgornji
implementaciji bisekcije nadomestimo primerjavo elementa seznama s klicem
funkcije, in bo postopek deloval enako.
Paziti moramo, da to deluje le, če je funkcija \koda{f} naraščajoča, torej če za
vsaki možni vrednosti \koda{x} in \koda{y}, kjer je \koda{x <= y}, velja
\koda{f(x) <= f(y)}.

% LocalWords:  podseznama
