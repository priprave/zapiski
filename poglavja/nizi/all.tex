Poleg dela s številkami od računalnika pogosto želimo, da nekaj naredi z nizi
besedila.
Primeri takšnih programov so npr.~urejevalniki besedila, ki jih uporabljamo tako
za pisanje \enquote{enostavnega} besedila (kode), kot tudi za razna obogatena
besedila.
Pravzaprav pa skoraj vsak računalniški program dela z besedilom; kadarkoli
želimo uporabniku prikazati neke informacije, jih moramo namreč izpisati na
zaslon.
Ko smo delali s številkami, smo problem izpisovanja prepustili računalniku, ker
je kodo za branje in izpisovanje številk k sreči napisal že nekdo drug.
Za pisanje splošnih programov pa tovrstno znanje ne bo dovolj, zato si poglejmo
osnove dela z besedili.

Pri slovenščini se naučimo, da je besedilo sestavljeno iz več odstavkov,
odstavek iz več povedi, poved iz več stavkov, stavek iz več besed, besede pa iz
več črk.
Pri tem se moramo zavedati, da stavke ločimo z ločili (vejice, pike, klicaji,
itd.), besede ločimo s presledki, posamične odstavke pa ločimo z zamikanjem prve
povedi v desno.
Za predstavitev v računalniku je tak model preveč zakompliciran, zato vzamemo
bolj enostavnega.
Besedila bomo predstavili z \emph{nizi} (angl.~\emph{string}), ki bodo
zaporedja več \emph{znakov} (angl.~\emph{character}).
Vse, kar bi si kadarkoli zaželeli izpisati, bomo proglasili za znak.
Tako si bomo vse črke predstavljali kot znak, kjer bomo ločili tudi med velikimi
in malimi črkami (saj vendar izgledajo drugače, če jih napišemo), prav tako bomo
za znake proglasili tudi ločila, oklepaje in matematične operacije
(\verb|+, -, *, /|).
Poleg tega bomo za znak proglasili tudi števke od 0 do 9, ker tudi njih
izpišemo (večje številke pa so sestavljene iz teh števk, zato ne potrebujemo
posebnih znakov za njih).

Nenazadnje bomo ustvarili še nekaj posebnih znakov, ki jih morda nebi
pričakovali.
Od teh bomo zdaj spoznali tri: znak za presledek, znak za novo vrstico in znak
za konec besedila.
Znak za presledek bomo uporabili, kjerkoli želimo imeti prostor med dvema
besedama (torej presledek).
Za razliko od slovenščine tudi presledke obravnavamo, kot da bi bili pravzaprav
neke posebne črke.
Znak za novo vrstico bomo uporabili tam, kjer želimo, da izpis našega programa
skoči vrstico nižje; brez tega znaka nam bo program vse izpisal v eni zelo dolgi
vrstici besedila.
Ta znak označimo s posebno kodo \verb+\n+ (ker se v angleščini ta znak imenuje
\emph{new line}), opazimo pa, da ga v funkciji \koda{printf} uporabljamo že od
prvega programa, ki smo ga napisali.
Uporabili bomo tudi znak za konec besedila, ki ga označimo z \verb+\0+,
pogosto pa mu rečemo tudi \emph{NULL}.
Več o temu znaku bomo povedali kasneje.

\naslov{Predstavitev znakov}

Preden si pogledamo nize, moramo razumeti, kako delamo z znaki.
V C++-u imamo za to poseben tip \koda{char}, ki nam hrani en znak.
Če želimo spremenljivki tipa \koda{char} nastaviti vrednost, moramo želeni znak
dati v enojne narekovaje, kot spodaj:

\fkoda{poglavja/nizi/basic-char-syntax.cpp}

S to kodo ustvarimo spremenljivko tipa \koda{char}, ki hrani vrednost
\koda{'A'}, torej znak za veliko črko A.
Tako kot števila lahko tudi znake pišemo in beremo; pri tem uporabimo
\koda{printf} s formatnikom \verb+%c+, narejen za znake.

\fkoda{poglavja/nizi/scanf-printf-char.cpp}

\begin{table}[h!]
  \centering
  \begin{tabular}[h!]{|c|c|}
	\hline
	Znak & ASCII koda \\
	\hline
	NULL (\verb+\0+) & 0 \\
	Nova vrstica (\verb+\n+) & 10 \\
	Presledek (\verb+' '+) & 32 \\
	\hline
	\verb+0+ & 48 \\
	\verb+1+ & 49 \\
	\verb+2+ & 50 \\
	\vdots & \vdots \\
	\verb+9+ & 57 \\
	\hline
	\verb+A+ & 65 \\
	\verb+B+ & 66 \\
	\verb+C+ & 67 \\
	\vdots & \vdots \\
	\verb+Z+ & 90 \\
	\hline
	\verb+a+ & 97 \\
	\verb+b+ & 98 \\
	\verb+c+ & 99 \\
	\vdots & \vdots \\
	\verb+z+ & 122 \\
	\hline
  \end{tabular}
  \caption{Del ASCII tabele}%
  \label{tab:ascii}
\end{table}

Ker so računalniki narejeni za delo s številkami, moramo tudi znake
predstaviti kot številke.
To dosežemo s t.i.~\emph{kodnimi tabelami}, ki vsakemu znaku priredijo eno
številko.
Najpreprostejša kodna tabela je ASCII, ki lahko zakodira vse črke angleške
abecede ter vse ostale zgoraj naštete znake, ne zmore pa zakodirati šumnikov ali
večine črk, ki se ne pojavljajo v angleški abecedi.
Prav zaradi tega razloga se pri programiranju takih črk izogibamo, kar se le da.
ASCII kode nekaterih pogostih znakov so prikazane v tabeli~\ref{tab:ascii}.
Opazimo lahko, da so števke in črke v tabeli zaporedno; števka \verb+0+ ima kodo
48, števka \verb+1+ 49, \ldots, črka \verb+A+ ima kodo 65, \verb+B+ ima kodo 66,
itd.
Opazimo tudi, da so velike črke od malih ločene, in da imajo male črke večje
kode.

Ta dejstva lahko uporabimo v programih tako, da črke preprosto obravnavamo, kot
da bi bile številke.
Črki \koda{'a'} lahko npr.~prištejemo neko številko, in tako dobimo črko, ki je
toliko znakov naprej v abecedi; \koda{'a' + 7} je na primer enako \koda{'h'}.
Poleg tega lahko znake med sabo primerjamo, kar bomo videli v prvem primeru.

Poglejmo si spodnji primer, kjer je prikazano, kako na enostaven način
preverimo, ali je neka črka velika ali majhna.
Koda sprva prebere eno črko iz vhoda, nato pa preveri, če je vpisana črka med
\koda{A} in \koda{Z}; če ni, potem preverimo še, ali je črka med \koda{a} in
\koda{z}.

\fkoda{poglavja/nizi/is-capital-letter.cpp}

Če dobro pogledamo v tabelo, vidimo, da koda $0$ ne pripada števki \koda{0}, pač
pa znaku za konec besedila.
To se morda na prvi pogled zdi nepričakovano, ampak ima svoj smisel; če je
števka del besedila, o njej ne razmišljamo kot o številki, temveč pač o nekem
znaku, ki ima v drugem kontekstu drugačen pomen.
Če želimo to števko pretvoriti v številko, s katero lahko brez skrbi računamo,
lahko uporabimo trik, kjer \enquote{odštejemo nič}, kot spodaj:

\fkoda{poglavja/nizi/odstevanje-nic.cpp}

Pri tem triku moramo biti previdni, da odštejemo pravilno ničlo; če odštejemo
številko $0$, se ne bo nič spremenilo; tako kot pri matematiki namreč
odštevanje ničle številke ne spremeni.
Če pa odštejemo \emph{znak} \koda{'0'} (v enojnih narekovajih), pa dejansko
odštevamo številko $48$, tj.~ASCII kodo znaka \koda{'0'}.
Praktično uporabo tega trika bomo pokazali v naslednjem razdelku.

\naslov{Predstavitev nizov}

Niz predstavimo kot seznam znakov, ki ga pišemo podobno kot seznam številk:

\fkoda{poglavja/nizi/basic-string-syntax.cpp}

Ta ukaz pove računalniku, naj ustvari spremenljivko z imenom
\koda{niz_besedila}, ki hrani \emph{največ} $300$ znakov.
V tej spremenljivki bomo hranili naše zaporedje besedila.
Če želimo nize brati ali pisati, uporabimo funkciji, ki ju že poznamo, ter
formatnik \verb+%s+, tu pa je ena posebnost; za branje nizov pred imenom
spremenljivke \underline{ne} zapišemo znaka \verb+&+, kakor to zapišemo za
branje številk ali znakov.
Če želimo neko spremenljivko nastaviti na niz, ki ga ne bomo prebrali, jo lahko
nastavimo na običajen način z enačajem, ter z dvojnimi narekovaji.
Tak način podajanja nizov smo že srečali; namreč vedno, ko uporabimo funkciji
\koda{scanf} ali \koda{printf}.

Poglejmo si preprost primer uporabe.
Spodnji program prebere uporabnikovo ime in ga pozdravi.

\fkoda{poglavja/nizi/pozdravljen.cpp}

Ko ustvarimo niz, računalniku povemo, kolikšna je njegova najdaljša možna
dolžina.
Nič pa nam ne preprečuje, da v to spremenljivko shranimo krajši niz.
Kako pa potem računalnik ve, kje se naš niz dejansko konča?
Pričakujemo namreč, da bomo za zapis kratkega niza uporabili nekaj mest za znake
na začetku, potem pa se bo niz nekje končal; kaj je na neuporabljenih mestih
zaporedja, nas ne zanima.
Ravno iz tega razloga so nizi zgrajeni tako, da imajo na koncu dodaten znak, ki
označuje konec besedila.
To je znak NULL, ki smo ga omenili na začetku.
Ta znak pove računalniku, da se besedilo tu konča in da naj naprej ne gleda.
Če vrednost niza nastavimo z enačajem ali niz preberemo s \koda{scanf}, bo
računalnik sam poskrbel, da bo ta znak napisan na pravo mesto; če pa z nizi
delamo kaj bolj zapletenega, moramo za ta znak skrbeti sami.
Zaradi tega znaka je dejansko število vidnih znakov, ki jih lahko shranimo v
niz, za eno manjše od predpisane največje dolžine.
Da se tovrstnim problemom izognemo, bomo od sedaj naprej vedno napisali nekaj
večjo število za dolžino niza; če pričakujemo, da uporabnik vpiše največ $200$
znakov, bomo za velikost niza dejansko napisali $201$ (ali celo malo več).

Kot primer, kako uporabimo zgornje dejstvo, si poglejmo spodnji program, ki
izračuna dolžino niza.

\fkoda{poglavja/nizi/dolzina-niza.cpp}

Ta koda ni težka, vendar jo je pogosto neprijetno pisati, zato imamo boljšo
alternativo; če na začetek programa dodamo \verb+#include <string.h>+, lahko
uporabljamo funkcijo \koda{strlen}, ki nam ravno tako izračuna dolžino niza:

\fkoda{poglavja/nizi/dolzina-niza-bolje.cpp}

Funkcijo \koda{strlen} uporabljamo v skoraj vsakem programu z nizi, zato je
dobro, da se je čim prej navadimo.
Poleg tega \koda{strlen} dolžino dejansko izračuna hitreje kakor zgornja koda.

V naslednjem primeru bomo napisali kodo, ki pretvori besedilo v velike črke.
Za to uporabimo eno od lastnosti ASCII tabele, ki smo jo omenili prej; namreč,
da so črke napisane zaporedno, in da so velike črke pred malimi.

\fkoda{poglavja/nizi/upcase.cpp}

Za zadnji primer v tem razdelku si poglejmo, kako bi pretvorili številko,
zapisano z nizom, v številko, zapisano v spremenljivki tipa \koda{int}.
Prej omenjen trik z odštevanjem nič ne bo deloval, ker lahko z njim
pretvarjamo le znake; lahko pa številko pretvorimo znak po znak.
Pri tem pretvarjanju uporabljamo lastnosti desetiškega zapisa števil; namreč,
da zaporedna mesta v zapisu predstavljajo vrednosti, ki se razlikujejo za
faktor $10$.
Ko pretvorimo prvi del besedila, in želimo dopisati še eno števko, moramo že
zapisani del \enquote{premakniti} eno mesto v levo, ter premaknjenemu številu
prišteti novo števko.
Premikanje dosežemo z množenjem z $10$.

\fkoda{poglavja/nizi/string-v-stevilo.cpp}

\naslov{Standardne funkcije}

Izkaže se, da pri delu z nizi pogosto pišemo zelo podobne kose programa, kakor
se je zgodilo pri primeru z izračunom dolžine.
Namesto da večkrat napišemo skoraj enako kodo, so v knjižnici \verb+string.h+
dostopne razne funkcije, ki nam pogosto olajšajo delo.

\podnaslov{Primerjava nizov}

Za primerjavo dveh nizov \underline{ne} uporabljamo dvojnega enačaja
(\koda{==}), temveč funkcijo \koda{strcmp}.
Funkcijo uporabimo tako, da ji v okrogle oklepaje napišemo dva niza;
\koda{strcmp(niz1, niz2)}.
Če sta niza enaka, funkcija vrne rezultat \texttt{0}, sicer pa vrne drugačen
rezultat.
Spodnja koda preveri, če je uporabniku ime Filip:

\fkoda{poglavja/nizi/ali-si-filip.cpp}

Funkcija nam pravzaprav poda več informacij.
Z njo lahko pogledamo, kakšna je \emph{leksikografska ureditev} dveh nizov;
preprosto povedano, kateri od nizov bi se, če bi bila oba niza besedi, pojavil
prej v slovarju (leksikonu).
Če bi se prvi niz pojavil prej, funkcija vrne negativno število.
Če bi se drugi niz pojavil prej, pa funkcija vrne pozitivno število.

\podnaslov{Kopiranje nizov}

Če želimo eno spremenljivko prekopirati v drugo, lahko napišemo \koda{b = a}.
Na žalost pa to ne deluje za nize; namesto \koda{niz2 = niz1} moramo napisati
\koda{strcpy(niz2, niz1)}.
Funkcija \koda{strcpy} prekopira drugi niz v prvega; na koncu bosta oba niza
imela enako vsebino.

Če želimo nekemu nizu na konec dodati nek drug niz, lahko za to uporabimo
funkcijo \koda{strcat} (\emph{string concatenate}).
Ta funkcija prav tako sprejme dva niza; ko jo pokličemo, drug niz kopira na
konec prvega.

\podnaslov{Operacije na prvih $m$ znakih}

Včasih želimo kopirati ali primerjati le del niza.
Za to imamo na voljo malce drugačne verzije zgoraj naštetih funkcij; če v imenih
teh funkcij za \koda{str} dodamo še \koda{n} (torej \koda{strncmp},
\koda{strncpy}, \ldots), in funkciji kot zadnji argument podamo številko $m$, bo
funkcija svoje delo opravila le na prvih $m$ znakih;
\koda{strncmp(niz1, niz2, 3)} bo primerjal le prve tri znake,
\koda{strncpy(niz2, niz1, 7)} bo kopiral le prvih sedem znakov, ipd.