Pogosto želimo, da računalnik izvaja drugačno kodo glede na vrednost ene ali
več spremenljivk, npr.~da nam pokaže drugačno vsebino, če smo napisali
pravilno ali napačno geslo, da računalo sešteva, če smo pritisnili gumb za
seštevanje, oz.~odšteva, če smo pritisnili gumb za odštevanje.
Z drugimi besedami, želimo upravljati potek programa (torej izbrati, katera koda
naj se izvede) glede na vrednosti spremenljivk.
Angleško takemu upravljanju pravimo \emph{control flow}, najpogosteje pa ga
izvajamo s t.i.~\emph{pogojnimi stavki}.
Osnovna struktura je sledeča:

\fkoda{poglavja/pogojni-stavki/sintaksa.cpp}

\emph{Pogoj} je nov pojem. Označuje neke vrste račun, katerega rezultat ni
število, vendar \emph{logična vrednost}.
Tu sta možni vrednosti le dve: pravilno (angl.~\koda{true}) in napačno
(angl.~\koda{false}).
Če bo rezultat računa, navedenega v običajnih oklepajih v zgornjem \koda{if}
stavku, \koda{true}, se bo izvedla koda znotraj prvih zavitih oklepajev, če pa
je rezultat računa \koda{false}, pa se bo izvedla koda v drugih zavitih
oklepajih (tistih za besedo \koda{else}).
Drugega dela, tj.~\koda{else} in oklepaje za njim, ni treba pisati, če tega ne
želimo.

Kako pa zapišemo pogoj?
Za to uporabimo posebne \emph{logične operatorje}.
Pri delu s številkami so nam na voljo naslednji:
\begin{itemize}
\item \koda{==}: primerja dve številski vrednosti.
  Rezultat je \koda{true}, če sta vrednosti enaki.
\item \koda{!=}: primerja dve številski vrednosti.
  Rezultat je \koda{true}, če sta vrednosti različni.
\item \koda{<}: primerja dve številski vrednosti.
  Rezultat je \koda{true}, če je vrednost na levi manjša od vrednosti na desni.
\item \koda{>}: deluje podobno kot \koda{<}, le da v drugo smer;
  rezultat je \koda{true}, če je vrednost na desni manjša od vrednosti na levi.
\item \koda{<=}: primerja dve številski vrednosti.
  Rezultat je \koda{true}, če sta vrednosti enaki, ali če je vrednost
  na levi manjša od vrednosti na desni.
\item \koda{>=} deluje podobno kot \koda{<=}, le da v drugo smer.
\end{itemize}

Poglejmo si primer uporabe pogojnega stavka.

\fkoda{poglavja/pogojni-stavki/stevila.cpp}

Program v zgornjem primeru primerja dve števili z vsemi naštetimi operatorji.
Če v program vpišemo npr.~števili $3$ in $7$, vstopimo v drugi, tretji in peti
pogojni stavek, zaradi česar se izpišejo naslednje vrstice:
%
\begin{verbatim}
Stevili sta razlicni.
Prvo stevilo je manjse od drugega.
Prvo stevilo je manjse ali enako drugemu.
\end{verbatim}
%
Če pa v program dvakrat vnesemo število $12$ (ali katerokoli drugo število), pa
dobimo naslednji izhod:
%
\begin{verbatim}
Stevili sta enaki.
Prvo stevilo je manjse ali enako drugemu.
Prvo stevilo je vecje ali enako drugemu.
\end{verbatim}

Pozorni moramo biti, da pri primerjavi enakosti dveh števil napišemo dva
enačaja, \koda{==}.
Če zapišemo le en enačaj, kot v matematiki (torej \koda{=}), se bo program sicer
zagnal, vendar ne bo deloval pravilno.
Enojnega enačaja nikoli ne uporabljamo v pogoju \koda{if} stavka!

Oglejmo si še primer uporabe stavka \koda{else}.
Spodnji program bo uporabnika vprašal za PIN, in mu napisal, če je bil PIN
pravilen oziroma napačen.

\fkoda{poglavja/pogojni-stavki/pin.cpp}

Pogojne stavke lahko tudi gnezdimo, torej vstavimo enega v drugega.
Spodnji program od uporabnika sprejme naročilo v restavraciji, kjer ponujajo
dve vrsti hrane; juhe in sendviče.
Na voljo sta dve vrsti juhe, in dve vrsti sendvičev.
Za izbiro kosila uporabnik prvo izbere med juho in sendvičem, nato pa še okus.

\fkoda{poglavja/pogojni-stavki/nesting.cpp}

Pozorni bodimo na postavitev kode.
Običajno kodo znotraj zavitih oklepajev \koda{if} stavka pišemo tako, da je
poravnana štiri presledke bolj desno od kode zunaj \koda{if} stavka.
V nekaterih programskih jezikih je taka poravnava obvezna, v C/C++ pa ne, vendar
nezamaknjena koda že v majhnih programih postane popolnoma nepregledna.
Branje in popravljanje kode je veliko lažje, če del kode znotraj zavitih
oklepajev zamaknemo za štiri presledke.
Za to lahko uporabimo tudi tipko Tab, ki se na tipkovnici nahaja levo od tipke
Q.
Kode kot spodaj nikoli ne pišemo!

\fkoda{poglavja/pogojni-stavki/unindented.cpp}

% LocalWords:  nezamaknjena
