\naslov{Sintaksa}

Ko si želimo podatke shraniti tako, da jih bomo lahko spreminjali in na koncu
nekaj z njimi naredili (npr.~da z njimi računamo, jih izpišemo, itd.), za to
uporabimo spremenljivke.
Vsaka spremenljivka hrani en podatek -- vse spremenljivke do sedaj so hranile le
eno številko.
Pogosto pa si želimo števila shraniti tako, da bomo kasneje lahko dostopali do
njih, ampak med pisanjem programa ne vemo točno, s koliko števili bo program
moral delati.
Problem rešimo s seznami.

Da ustvarimo seznam (angl.~\emph{array}), zapišemo tip spremenljivke, ki ga bodo
imeli vsi elementi seznama (trenutno poznamo le \koda{int}, a bomo kmalu
spoznali tudi druge tipe spremenljivk), nato seznamu damo ime, in na koncu v
oglatih oklepajih zapišemo dolžino seznama, takole:

\fkoda{poglavja/seznami/sintaksa.cpp}

Tukaj smo ustvarili seznam z imenom \koda{seznam_stevil}, ki hrani $300$ števil.
Če želimo dostopati do elementov seznama, ali nastaviti njihove vrednosti, tudi
uporabimo oglate oklepaje, kot v spodnjem primeru.

\fkoda{poglavja/seznami/dostopanje.cpp}

Tu smo prvo nastavili tretji element seznama na $7$, nato smo nastavili četrti
element seznama na $9$, in za tem nastavili peti element seznama na njuno vsoto.
Ostalih elementov seznama se nismo dotaknili.
Tako kot pri spremenljivkah nismo smeli uporabiti vrednosti spremenljivke,
preden smo ji vrednost nastavili, moramo paziti, da vrednosti posamičnega
elementa seznama ne uporabljamo, preden je ne nastavimo.
Narobe bi bilo na primer izpisati \koda{seznam_stevil[2]} ali to vrednost
uporabiti v računu, saj je nismo nikoli nastavili.

Pri dostopanju do elementov seznama moramo paziti tudi na naslednje dejstvo:
če imamo seznam dolžine $N$, potem so elementi tega seznama oštevilčeni s
številkami od $0$ do $N-1$ vključno, ne pa s številkami od $1$ do $N$, kot bi
morda pričakovali.
V zgornjem primeru tako lahko zapišemo \koda{seznam_stevil[0]},
\koda{seznam_stevil[1]}, \ldots, \koda{seznam_stevil[299]}, ne pa tudi
\koda{seznam_stevil[300]}.
Pravimo, da so seznami \emph{indeksirani} od $0$ naprej.

Poglejmo si primer preproste naloge.
Na vhodu je podano število $N$, ki mu sledi $N$ števil.
Program mora izpisati ta števila (razen prvega, $N$), v obratnem vrstnem redu.
V ta namen ustvarimo seznam z imenom \koda{seznam}, ki lahko shrani največ
$1000$ (predpostavimo, da uporabnik ne bo vnesel več kot $1000$ števil).
Potem s \koda{for} zanko preberemo $N$ števil in vsako shranimo v svoj element
seznama, začenši z $0$.
Na koncu se s še eno \koda{for} zanko sprehodimo skozi $i = 0, 1, \ldots, N-1$.
Na vsakem koraku izračunamo indeks (položaj) elementa, ki je $i$-ti od konca
seznama, in ga shranimo v spremenljivko \koda{obratni}.
Ta indeks je natanko \koda{N-i-1}, saj mora za \koda{i=0} biti
\koda{obratni=N-1}, za \koda{i=1} mora biti \koda{obratni=N-2}, itd.
Da se seznam na izhodu izpiše v obratnem vrstnem redu, preprosto izpišemo
\koda{seznam[obratni]}.

\fkoda{poglavja/seznami/obrni.cpp}

Poglejmo si še malo bolj zanimiv primer.
Naša naloga sedaj je, da uredimo seznam $N$ števil ($N \le 10^6$) po velikosti
od najmanjšega do največjega.
Podano imamo tudi informacijo, da bodo ta števila velika med vključno $0$ in
$100$.
Naloge se lahko lotimo tako, da preštejemo, kolikokrat se neko število pojavi v
danem seznamu, nato pa bomo seznam rekonstruirali tako, da bo urejen.
Da preštejemo, kolikokrat se kakšno število pojavi, uporabimo nov seznam, kjer
indeks pomeni številko, ki jo štejemo, shranjena vrednost pa kolikokrat smo to
številko že prešteli.

\fkoda{poglavja/seznami/counting.cpp}

Ta algoritem za urejanje je zelo znan; imenuje se urejanje s preštevanjem
(angl.~\emph{counting sort}).
Primeren je, kadar imamo zelo majhen razpon možnih vrednosti števil, kakor smo
imeli tu ($0 - 100$).
Obstajajo tudi drugi algoritmi za urejanje.
Nekatere bomo spoznali kasneje.

\podnaslov{Večdimenzionalni seznami}

Videli smo, kako ustvariti seznam števil, kaj pa seznam seznamov?
Takemu seznamu pravimo \emph{dvodimenzionalen seznam}, ustvarimo pa ga tako,
da napišemo dva zaporedna oglata oklepaja z velikostjo, kot spodaj:

\fkoda{poglavja/seznami/sintaksa-2d.cpp}

Dvodimenzionalni seznami so uporabni, kadar moramo podatke predstaviti v tabeli.
Do posamičnih elementov dostopamo z dvojnimi oglatimi oklepaji, tako kot pri
inicializaciji spremenljivke; \koda{tabela[i][j]}.
Tabele si običajno predstavljamo tako, da nam prvi indeks poda zaporedno
številko vrstice, drugi pa zaporedno številko stolpca.
Sicer pa z njimi delamo enako kot z običajnimi seznami.