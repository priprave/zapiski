\naslov{Računalniški spomin}

Spomin, pomnilnik ali angl.~RAM (\emph{Random access memory})
je ključna komponenta v računalniku. Med tekom programa so v njem shranjene vse
spremenljivke, ki jih program uporablja.
S spremenljivkami lahko naredimo tudi več, kot smo se do sedaj naučili, vendar
pa moramo za to razumeti, kako je spomin zgrajen.

Osnovna enota za merjenje količine informacij je \emph{bit}.
En bit informacij ustreza odgovoru na eno vprašanje tipa da ali ne
-- če nam nekdo pove, da so vrata zaprta, nam je podal en bit informacij,
ker so lahko vrata bodisi odprta bodisi zaprta.
Če imamo v omari dva para hlač, dve majici in dve kapi, lahko opišemo,
kako smo oblečeni, s tremi biti informacije --- za vsak kos oblačila porabimo
en bit.

V računalništvu bite najpogosteje označujemo z ničlami in enicami.
Običajno ničla predstavlja odgovor \enquote{ne} na dano vprašanje,
enica pa odgovor \enquote{da}.
Bit pa je zelo majhna količina informacije, zato pogosto govorimo v večjih
enotah, kot so \emph{bajti}, \emph{kilobajti}, \emph{megabajti} itd.
En bajt ustreza osmim bitom, kilobajt je tisoč bajtov, megabajt je tisoč
kilobajtov, gigabajt je tisoč megabajtov, in terabajt je tisoč gigabajtov.

Računalniški spomin je sestavljen iz spominskih celic, ki so dolge en bajt.
Vsaka od teh celic ima svoj \emph{naslov} --- številko, s katero lahko to
celico ločimo od ostalih.
Naslovi so zaporedne številke od $0$ do velikosti pomnilnika, ki ga imamo
nameščenega v računalniku.
Celice so naraščajoče urejene po svojih naslovih, tako da je celica številka
$150337$ sosednja celicama s številkama $150336$ in $150338$.

Upravljanje z računalniškim spominom je ena od nalog operacijskega sistema.
Naši programi operacijski sistem med izvajanjem prosijo za neko količino spomina,
operacijski sistem pa določi, katere spominske celice bo program prejel.
Te celice tedaj pripadajo programu, dokler se ta ne zaključi, ali dokler tega
spomina ne vrne operacijskemu sistemu na drugačen način.
Med izvajanjem našega programa praviloma noben drug program nima dostopa do tega
dela spomina.

Kako pa program ve, koliko spomina bo potreboval?
Da to izračuna, se zanaša na tipe.
Vsaka spremenljivka ima tip, vsak tip pa ima fiksno dolžino, ki jo zavzame v
spominu.
Dolžine pogostih tipov so sledeče:
\begin{itemize}
\item \koda{int}: $4$ bajti
\item \koda{long long}: $8$ bajtov
\item \koda{char}: $1$ bajt
\item \koda{bool}: $1$ bajt
\end{itemize}
Spremenljivke, ki v spominu zavzamejo več kot $1$ bajt, moramo shraniti v več
kot eno spominsko celico.
Celice, v katere te vrednosti zapišemo, so v spominu zaporedne; če imamo
spremenljivko tipa \koda{int}, bo tako zavzela $4$ zaporedne celice.

\naslov{Kazalci}

V tem razdelku spoznamo kazalce, ki so pomemben koncept v programiranju, pred
tem pa podajmo še eno opombo.
Razumevanje kazalcev je ključno za razumevanje bolj zapletenih podatkovnih
struktur ter nekaterih algoritmov, vendar se lahko z njihovo uporabo hitro
zmotimo.
Poleg tega je razhroščevanje kode z veliko kazalci pogosto zelo zapleteno, zato
se v tekmovalnem programiranju direktne uporabe kazalcev vestno izogibamo.
Pogosto jih lahko nadomestimo z indeksi v seznamu ali drugačno strukturo kode, s
čimer skoraj vedno pridobimo na razumljivosti kode.
Kazalce na tekmovanjih uporabimo le, če je to res nujno!

Ker so spominski naslovi številke, jih lahko shranjujemo, kakor shranjujemo
ostale številke.
Za to imamo v C++ na voljo poseben tip, ki mu rečemo \emph{kazalec}
(angl.~\textit{pointer}).
Pravzaprav kazalec ni sam svoj tip, ampak razširitev nekega drugega tipa;
pravimo, da kazalec \emph{kaže na drug tip}.
Da ustvarimo nov kazalec, zapišemo ime tipa, na katerega želimo kazati,
nato pa pred ime spremenljivke damo zvezdico~\verb+*+.
Kazalcu lahko nastavimo vrednost tako, da vanj shranimo naslov neke
spremenljivke, ki smo že ustvarili.
Do tega naslova dostopamo z operatorjem~\verb+&+.
Da dostopamo do vrednosti, shranjene v celici, na katero kazalec kaže,
uporabimo operator \verb+*+ (ki ima drugačen pomen, kot zvezdica v deklaraciji
spremenljivke).

\fkoda{poglavja/spomin-in-kazalci/sintaksa.cpp}

Vse, kar smo zgoraj delali s kazalci, je bilo možno (in lažje) narediti tudi z
običajnimi spremenljivkami.
Kazalci, ki kažejo na spremenljivke, ki tako in tako že obstajajo, so bolj ali
manj neuporabni.
Kako pa naredimo kazalec, ki kaže na del spomina, v katerem še ni nobene
spremenljivke?
Če želimo storiti kaj takega, moramo prevzeti odgovornost za upravljanje spomina
v našem programu.
Do sedaj je za to skrbel prevajalnik, ki je v naš program na pravilna mesta
zapisal ukaze, ki si spomin sposojajo od operacijskega sistema, ter ga vračajo,
ko ga ne potrebujemo več.
Bolj natančno; ko smo deklarirali spremenljivko, je prevajalnik poskrbel, da
prosimo za natanko toliko spomina, kolikor ga za to spremenljivko potrebujemo
(zato moramo za vsako spremenljivko zapisati tip), ter si njegov naslov
zapomnil, ko pa spremenljivke nismo več potrebovali, je prevajalnik poskrbel, da
ta del spomina vrnemo operacijskemu sistemu; temu pravimo \emph{sprostitev}
(angl.~\textit{deallocation}).

Za bolj sofisticirano uporabo spomina moramo ti vlogi prevzeti mi.
Za to sta nam na voljo dve funkciji: \koda{malloc} in \koda{free}.
Da ju uporabljamo, moramo vključiti knjižnico \koda{stdlib.h}.
Oblika funkcij je naslednja:
\fkoda{poglavja/spomin-in-kazalci/opis-malloc-free.cpp}
V obliki je posebnost, ki je še nismo omenili; ni namreč nujno, da ima vsak
kazalec tip.
Lahko imamo kazalce, ki kažejo na del spomina, mi pa (še) ne vemo, ali pa ni
pomembno, kaj je v tistem delu spomina shranjeno.
Za take kazalce pravimo, da kažejo na \koda{void}, kar pa ne pomeni, da ne
kažejo na nič; na lokaciji v spominu, kamor kažejo, je nekaj shranjeno, mi samo
ne vemo, kako naj te podatke interpretiramo.

Funkcija \koda{malloc} sprejme en argument, in vrne kazalec na \koda{void}.
Argument je tipa \koda{size_t}, ki je za naše potrebe skoraj enak tipu
\koda{unsigned long long}, to je torej številka.
Argument pove, koliko bajtov spomina si želimo sposoditi od operacijskega
sistema.
Funkcija \koda{malloc} nato vrne kazalec na prvi naslov znotraj bloka spomina,
ki smo si ga ravno sposodili.
Ker operacijski sistem ne ve, kaj bomo v ta spomin shranili, nam \koda{malloc}
vrne \koda{void*}, mi pa ga moramo pretvoriti v pravi tip kazalca.
To storimo tako, da tik pred klic funkcije v oklepaje zapišemo želeni tip, kakor
bomo videli v naslednjem primeru.

Funkcija \koda{free} je ravno nasprotna od \koda{malloc}.
Sprejme kazalec, ki ga nam je dal \koda{malloc}, ter sprosti del spomina, na
katerega kazalec kaže.
Le-ta bo po klicu \koda{free} še vedno obstajal, in bo še vedno kazal na isto
mesto.
Edina sprememba je, da del spomina, na katerega kaže, ne pripada več našemu
programu, in ga ne smemo uporabljati.

\fkoda{poglavja/spomin-in-kazalci/dinamicno-upravljanje-spomina.cpp}

V primeru zgoraj smo uporabili operator \koda{sizeof}, ki nam preprosto pove,
koliko bajtov zasede naveden tip.

\naslov{Kako delujejo seznami}

Nič nas ne omejuje, da od operacijskega sistema zahtevamo zelo velik blok
spomina, tudi po več sto tisoč bajtov.
Pa imamo lahko kakšen utemeljen razlog, da si toliko spomina izposodimo?
Da, ravno to stori prevajalnik, ko ustvarimo seznam.
Poglejmo si, kako ustvarimo seznam samo s kazalci.

\fkoda{poglavja/spomin-in-kazalci/seznam.cpp}

V zgornjem primeru uporabljamo dva nova operatorja na kazalcih; seštevanje in
oglate oklepaje.
Če kazalcu \koda{seznam} prištejemo število \koda{i}, dobimo
nov kazalec, ki kaže na mesto \koda{seznam + (velikost tipa) * i}, torej na
idealno mesto, kamor lahko zapišemo \koda{i}-ti element seznama, če jih
zapisujemo enega za drugega.
Drug novi operator so oglati oklepaji --- ti se obnašajo popolnoma enako kot
v seznamih.
Oglati oklepaj \koda{seznam[i]} je pravzaprav krajšava za zapis
\koda{*(seznam+i)}, torej za dostop do \koda{i}-tega elementa v bloku spomina.

Tudi seznami, kakor smo jih spoznali prej, so dejansko kazalci na blok spomina,
le da s tem spominom upravlja prevajalnik.
Trik s prištevanjem števila k kazalcu deluje tudi za prištevanje števila k
seznamu.

\naslov{Podajanje po referenci}

Opazimo, da smo operator \koda{&} že srečali, in sicer čisto na začetku.
Pri branju številk iz vhoda moramo v \koda{scanf} zapisati ta operator pred
imenom spremenljivke.
Sedaj razumemo, zakaj je temu tako; \koda{scanf} sprejme kazalce na
spremenljivke, ki jih želimo prebrati, ter popravi vrednosti, na katere kažejo
kazalci, s prebranimi vrednostmi.

Funkcije, kot smo jih pisali do sedaj, niso sposobne spremeniti vrednosti
spremenljivk zunaj funkcije.
Spodnji program se na primer ne bo niti prevedel:
%
\fkoda{poglavja/spomin-in-kazalci/lokalne-spremenljivke.cpp}
%
Naslednji program pa se bo prevedel, vendar bo izhod morda v nasprotju s
pričakovanji:
%
\fkoda{poglavja/spomin-in-kazalci/lokalne-spremenljivke-2.cpp}

Spremenljivke, ki jih deklariramo v funkciji, to je znotraj telesa funkcije,
ali pa v seznamu argumentov, so lokalne na to funkcijo --- zunaj nje sploh ne
obstajajo.
Če želimo, da funkcija popravi neko vrednost, ki jo uporabljamo tudi
zunaj funkcije, smo do sedaj lahko to naredili samo tako, da smo spremenljivko
naredili globalno, torej dostopno vsem funkcijam v kodi.
Kaj pa, če želimo neko spremenljivko na tak način deliti samo med dvema
funkcijama?

Da odgovorimo na to vprašanje, moramo razumeti, kako se argumenti podajajo v
funkcije.
Ko neko funkcijo pokličemo, se argumenti, ki jih funkciji podamo,
\emph{prekopirajo} v poseben del spomina, ki ji pripada.
Ko smo znotraj ene funkcije, ne poznamo imena spremenljivk v drugih funkcijah;
prav tako ne vemo, kje so te spremenljivke shranjene.
Nič pa nam ne preprečuje, da spreminjamo spomin, ki našemu programu pripada, pa
četudi smo ga rezervirali v drugi funkciji; razen tega, da ne vemo, kateri del
spomina je naš, in kateri ni.
Lahko si predstavljamo, da smo zabredli v spominsko džunglo, v kateri ne
prepoznamo prave poti do spremenljivk, ki jih želimo popraviti.
Če pa s seboj prinesemo zemljevid, bomo nenadoma to lahko naredili.
Ta zemljevid je kazalec.

Funkcija lahko brez težav sprejme kazalec kot argument.
Če to storimo, pravimo, da smo funkciji vrednost podali \emph{po referenci}
(angl.~\textit{pass by reference}), namesto da bi argument podali običajno,
čemur pravimo \emph{podajanje po vrednosti} (angl.~\textit{pass by value}).
Kazalec, ki smo ga podali, se bo še vedno prekopiral v del spomina, ki pripada
funkciji, vrednost, na katerega kazalec kaže, pa bo ostala tam, kjer je.
Tako lahko skozi kazalec spremenimo vrednosti spremenljivk zunaj funkcije.
Prvi primer od zgoraj bi lahko tako popravili na naslednji način:

\fkoda{poglavja/spomin-in-kazalci/pass-by-reference.cpp}

%#endblock
% LocalWords:  razhroščevanje
